(1, 0,with different methods A human-centered approach must be in part an empirical science, in-)

(2, 0,Chapter)

(1, 2,What Is AI?)

(3, 3,neurophysiological evidence into computational models)

(4, 0,Chapter)

(1, 4,The Foundations of Artificial Intelligence)

(5, 5,Schickard (1592–1635), although the Pascaline, built in 1642 by Blaise Pascal (1623–1662),)

(6, 0,Chapter)

(1, 6,The Foundations of Artificial Intelligence)

(7, 7,its mathematical development really began with the work of George Boole (1815–1864), who)

(8, 0,Chapter)

(1, 8,The Foundations of Artificial Intelligence)

(9, 9,GAME THEORY)

(10, 0,Chapter)

(1, 10,The Foundations of Artificial Intelligence)

(11, 11,action, and consciousness or, in the pithy words of John Searle (1992), brains cause minds)

(12, 0,Chapter)

(1, 1014,Operations/sec)

(1017, 1013,Memory updates/sec)

(1014, 12,The Foundations of Artificial Intelligence)

(13, 13,vented independently and almost simultaneously by scientists in three countries embattled in)

(14, 0,Chapter)

(1, 14,The Foundations of Artificial Intelligence)

(15, 15,count of the behaviorist approach to language learning, written by the foremost expert in)

(16, 0,Chapter)

(1, 16,The History of Artificial Intelligence)

(17, 17,each other as they wrote each instruction to make sure they agreed)

(18, 0,Chapter)

(1, 18,The History of Artificial Intelligence)

(19, 19,number of customers Tom gets?)

(20, 0,Chapter)

(1, 20,The History of Artificial Intelligence)

(21, 21,algorithms use better representations and have shown more success)

(22, 0,Chapter)

(1, 22,The History of Artificial Intelligence)

(23, 23,standing The problems included representing stereotypical situations (Cullingford, 1981),)

(24, 0,Chapter)

(1, 24,The History of Artificial Intelligence)

(25, 25,new scruffy idea is another question)

(26, 0,Chapter)

(1, 26,The History of Artificial Intelligence)

(27, 27,tated text and just the dictionary definitions of the two senses—“works, industrial plant” and)

(28, 0,Chapter)

(1, 28,Summary)

(29, 29,or work from an ideal standard?)

(30, 0,Chapter)

(1, 30,Exercises)

(31, 31,second, compare with the high-end computer described in Figure ?)

(32, 0,Chapter)

(1, 32,Exercises)

(33, 1,energy away from new ideas?)

(2, 33,choice of action for every possible percept sequence, we have said more or less everything)

(34, 34,Agents and Environments)

(35, 35,show later in this chapter that it can be very intelligent)

(36, 1,Chapter)

(2, 36,Good Behavior: The Concept of Rationality)

(37, 37,sequence and whatever built-in knowledge the agent has)

(38, 1,Chapter)

(2, 38,Good Behavior: The Concept of Rationality)

(39, 39,vast variety of environments)

(40, 1,Chapter)

(2, 40,The Nature of Environments)

(41, 41,applicability of each of the principal families of techniques for agent implementation First,)

(42, 1,Chapter)

(2, 42,The Nature of Environments)

(43, 43,previous decisions; moreover, the current decision doesn’t affect whether the next part is)

(44, 1,Chapter)

(2, 44,The Nature of Environments)

(45, 45,of the particular case but might not identify a good design for driving in general For this)

(46, 1,Chapter)

(2, 46,The Structure of Agents)

(47, 47,actions Section  explains in general terms how to convert all these agents into learning)

(48, 1,Chapter)

(2, 48,The Structure of Agents)

(49, 49,mounted brake light Unfortunately, older models have different configurations of taillights,)

(50, 1,Chapter)

(2, 50,The Structure of Agents)

(51, 51,agent is that it does not have to describe “what the world is like now” in a literal sense For)

(52, 1,Chapter)

(2, 52,The Structure of Agents)

(53, 53,utility the agent expects to derive, on average, given the probabilities and utilities of each)

(54, 1,Chapter)

(2, 54,The Structure of Agents)

(55, 55,that this is a good thing; the percept itself does not say so It is important that the performance)

(56, 1,Chapter)

(2, 56,The Structure of Agents)

(57, 57,property is that of being identical to or different from another black box The algorithms)

(58, 1,Chapter)

(2, 58,Summary)

(59, 59,AI, the concept was of peripheral interest until the mid-1980s, when it began to suffuse many)

(60, 1,Chapter)

(2, 60,Exercises)

(61, 61,a An agent that senses only partial information about the state cannot be perfectly rational)

(62, 1,Chapter)

(2, 62,Exercises)

(63, 2,dirty Can you come up with a rational agent design for this case?)

(3, 63,satisfying it Let us first look at why and how an agent might do this)

(64, 64,Problem-Solving Agents)

(65, 65,as our agent We apologize to Romanian readers who are unable to take advantage of this pedagogical device)

(66, 2,Chapter)

(3, 66,Problem-Solving Agents)

(67, 67,some author use RESULT(a, s) instead of RESULT(s, a), and some use DO instead of RESULT)

(68, 2,Chapter)

(3, 70,Bucharest)

(86, 68,Example Problems)

(69, 69,the general flavor of their formulations)

(70, 2,Chapter)

(3, 70,Example Problems)

(2, 0,Goal State)

(8, 71,attacked by the queen at the top left)

(72, 2,Chapter)

(3, 72,Example Problems)

(73, 73,awards, and so on)

(74, 2,Chapter)

(3, 74,Searching for Solutions)

(75, 75,map shown in Figure —has only 20 states As we discuss in Section , loops can cause)

(76, 2,Chapter)

(3, 76,Searching for Solutions)

(77, 77,state-space graph into the explored region and the unexplored region, so that every path from)

(78, 2,Chapter)

(3, 78,Searching for Solutions)

(8, 79,operations on a queue are as follows:)

(80, 2,Chapter)

(3, 80,Uninformed Search Strategies)

(81, 81,and failed the goal test Now, the shallowest goal node is not necessarily the optimal one;)

(82, 2,Chapter)

(3, 82,Uninformed Search Strategies)

(83, 1,Memory)

(4, 5,megabytes)

(106, 7,seconds)

(1014, 15,years)

(1016, 83,rather than when it is first generated The reason is that the first goal node that is generated)

(84, 2,Chapter)

(3, 98,Bucharest)

(211, 84,Uninformed Search Strategies)

(85, 85,turn (A recursive depth-first algorithm incorporating a depth limit is shown in Figure ))

(86, 2,Chapter)

(3, 86,Uninformed Search Strategies)

(87, 87,be of length 19 at the longest, so ℓ= 19 is a possible choice But in fact if we studied the)

(88, 2,Chapter)

(3, 88,Uninformed Search Strategies)

(89, 89,Four iterations of iterative deepening search on a binary tree)

(90, 2,Chapter)

(3, 90,Uninformed Search Strategies)

(91, 91,positive ϵ; c optimal if step costs are all identical; d if both directions use breadth-first search)

(92, 2,Chapter)

(3, 92,Informed (Heuristic) Search Strategies)

(93, -1,Iasi)

(193, 93,that A∗uses g + h instead of g)

(94, 2,Chapter)

(3, 328,Bucharest)

(0, 328,Timisoara)

(329, 373,Zerind)

(193, 252,Timisoara)

(374, 365,Arad)

(366, 94,Informed (Heuristic) Search Strategies)

(95, 95,node n′ on the optimal path from the start node to n, by the graph separation property of)

(96, 2,Chapter)

(3, 96,Informed (Heuristic) Search Strategies)

(97, 379,N)

(420, 97,expanded in Figure  even though it is a child of the root We say that the subtree below)

(98, 2,Chapter)

(3, 98,Informed (Heuristic) Search Strategies)

(99, 99,f-value of the best leaf in the forgotten subtree and can therefore decide whether it’s worth)

(100, 2,Chapter)

(3, 645,Sibiu)

(417, 446,Fagaras)

(417, 446,and expanding Fagaras)

(417, 100,Informed (Heuristic) Search Strategies)

(101, 101,the overhead of maintaining the frontier and the explored set)

(102, 2,Chapter)

(3, 102,Heuristic Functions)

(2, 0,Goal State)

(8, 103,solved at reasonable computational cost)

(104, 2,Chapter)

(3, 1,A∗(h2))

(1641, 104,Heuristic Functions)

(105, 105,sly” Thus, there is a tradeoff between accuracy and computation time for heuristic functions)

(106, 2,Chapter)

(3, 0,Goal State)

(54, 106,Heuristic Functions)

(107, 107,h(n) = 0 for goal states, but it is not necessarily admissible or consistent)

(108, 2,Chapter)

(3, 108,Bibliographical and Historical Notes)

(109, 109,and at that time drew the attention of the eminent mathematician Carl Friedrich Gauss, who)

(110, 2,Chapter)

(3, 110,Bibliographical and Historical Notes)

(111, 111,gorithms behave identically with admissible heuristics, but RBFS expands nodes in best-first)

(112, 2,Chapter)

(3, 112,Exercises)

(113, 113,climbable 3-foot-high crates)

(114, 2,Chapter)

(3, 114,Exercises)

(115, 115,violates the property)

(116, 2,Chapter)

(3, 116,Exercises)

(117, 117,c Uniform-cost search is a special case of A∗search)

(118, 2,Chapter)

(3, 118,Exercises)

(119, 3,implementing the heuristics and comparing the performance of the resulting algorithms)

(4, 119,network optimization, vehicle routing, and portfolio management)

(120, 120,Local Search Algorithms and Optimization Problems)

(121, 121,are defined in the text)

(122, 3,Chapter)

(4, 122,Local Search Algorithms and Optimization Problems)

(14, 123,roughly 21 steps for each successful instance and 64 for each failure)

(124, 3,Chapter)

(4, 124,Local Search Algorithms and Optimization Problems)

(125, 125,SEARCH)

(126, 3,Chapter)

(4, 126,Local Search Algorithms and Optimization Problems)

(127, 23,Mutation)

(32543213, 127,In (c), two pairs are selected at random for reproduction, in accordance with the prob-)

(128, 3,Chapter)

(4, 128,Local Search in Continuous Spaces)

(129, 129,tions in continuous spaces The literature on this topic is vast; many of the basic techniques)

(130, 3,Chapter)

(4, 130,Local Search in Continuous Spaces)

(131, 2,f(x1, y1, x2, y2, x3, y3) =)

(3, 131,Given a locally correct expression for the gradient, we can perform steepest-ascent hill climb-)

(132, 3,Chapter)

(4, 132,Searching with Nondeterministic Actions)

(133, 133,to global) minima)

(134, 3,Chapter)

(4, 0,Beyond Classical Search)

(4, 134,Searching with Nondeterministic Actions)

(135, 5,Suck)

(6, 7,GOAL)

(8, 6,GOAL)

(1, 4,LOOP)

(5, 4,LOOP)

(5, 0,Suck)

(1, 7,GOAL)

(4, 135,solution found is shown in bold lines)

(136, 3,Chapter)

(4, 136,Searching with Nondeterministic Actions)

(137, 5,Right)

(5, 137,will eventually work, or perhaps one has the wrong key (or the wrong room!) After seven or)

(138, 3,Chapter)

(4, 138,Searching with Partial Observations)

(139, 139,transferred directly from the underlying physical problem)

(140, 3,Chapter)

(4, 1,Beyond Classical Search)

(3, 140,Searching with Partial Observations)

(141, 0,S)

(7, 141,the belief state, consisting of the first few states examined, is also unsolvable In some cases,)

(142, 3,Chapter)

(4, 142,Searching with Partial Observations)

(3, 143,belief-state problem from an underlying physical problem and the PERCEPT function Given)

(144, 3,Chapter)

(4, 6,Beyond Classical Search)

(2, 144,Searching with Partial Observations)

(2, 6,Right)

(5, 145,initial belief state b consists of the set of all locations The the robot receives the percept)

(146, 3,Chapter)

(4, 146,Online Search Agents and Unknown Environments)

(147, 147,as they are received rather than waiting for the entire input data set to become available)

(148, 3,Chapter)

(4, 0,S)

(3, 148,Online Search Agents and Unknown Environments)

(149, 149,agent most recently entered the current state To achieve that, the algorithm keeps a table that)

(150, 3,Chapter)

(4, 150,Online Search Agents and Unknown Environments)

(151, 151,grid, the probability that the walk ever returns to the starting point is only about  (Hughes, 1995))

(152, 3,Chapter)

(4, 0,Beyond Classical Search)

(4, 152,Summary)

(153, 153,combines pairs of states from the population)

(154, 3,Chapter)

(4, 154,Bibliographical and Historical Notes)

(155, 155,pal difference is that the representations that are mutated and combined are programs rather)

(156, 3,Chapter)

(4, 156,Exercises)

(157, 157,e Genetic algorithm with population size N = 1)

(158, 3,Chapter)

(4, 158,Exercises)

(159, 159,overcoming two successive motion errors and still reaching the goal)

(160, 3,Chapter)

(4, 4,reaching (1, −1)?)

(5, 160,attracted much interest in the AI community)

(162, 4,Chapter)

(5, 1,entry fee of 1)

(2, 162,Optimal Decisions in Games)

(163, -1,X)

(0, 163,the initial state, then MAX’s moves in the states resulting from every possible response by)

(164, 4,Chapter)

(5, 2,D)

(3, 164,Optimal Decisions in Games)

(165, 165,the backed-up value of X is this vector The backed-up value of a node n is always the utility)

(166, 4,Chapter)

(5, 166,Alpha–Beta Pruning)

(167, 167,prune entire subtrees rather than just leaves The general principle is this: consider a node n)

(168, 4,Chapter)

(5, 2,Adversarial Search)

(2, 168,Alpha–Beta Pruning)

(169, 169,moves) gets you to within about a factor of 2 of the best-case O(bm/2) result)

(170, 4,Chapter)

(5, 170,Imperfect Real-Time Decisions)

(171, 171,function should be strongly correlated with the actual chances of winning)

(172, 4,Chapter)

(5, 172,Imperfect Real-Time Decisions)

(173, 173,deepening also helps with move ordering)

(174, 4,Chapter)

(5, 174,Imperfect Real-Time Decisions)

(175, 0,a     b    c    d    e     f     g    h)

(8, 175,evaluation function and a large database of optimal opening and endgame moves)

(176, 4,Chapter)

(5, 176,Stochastic Games)

(177, 0,possible moves)

(25, 177,there are only 21 distinct rolls The six doubles (1–1 through 6–6) each have a probability of)

(178, 4,Chapter)

(5, 0,B)

(1, 1,C)

(1, 178,Stochastic Games)

(179, 1,MAX)

(21, 179,It may have occurred to you that something like alpha–beta pruning could be applied)

(180, 4,Chapter)

(5, 180,Partially Observable Games)

(181, 181,this issue for now; Chapters 7 and 8 suggest methods for compactly representing very large belief states)

(182, 4,Chapter)

(5, 0,a)

(4, 182,Partially Observable Games)

(183, 12,cards each, so the number of deals is)

(13, 183,difficult, so solving ten million is out of the question Instead, we resort to a Monte Carlo)

(184, 4,Chapter)

(5, 0,a)

(1, 184,State-of-the-Art Game Programs)

(185, 185,aggressive use of the null move heuristic and forward pruning)

(186, 4,Chapter)

(5, 186,Alternative Approaches)

(187, 187,max, evaluation functions, and alpha–beta, is just one way to do this Probably because it has)

(188, 4,Chapter)

(5, 98,MAX)

(99, 188,Summary)

(189, 189,achieves much greater efficiency by eliminating subtrees that are provably irrelevant)

(190, 4,Chapter)

(5, 190,Bibliographical and Historical Notes)

(191, 191,et al (1998) describe several variants of Monte Carlo search, including one where MIN has)

(192, 4,Chapter)

(5, 192,Bibliographical and Historical Notes)

(193, 193,three games in all that time In the first match against CHINOOK, Tinsley suffered his fourth)

(194, 4,Chapter)

(5, 194,Exercises)

(195, 195,d Give an informal proof that someone will eventually win if both play perfectly)

(196, 4,Chapter)

(5, 196,Exercises)

(197, 0,B)

(2, 197,and crosses) as an example We define Xn as the number of rows, columns, or diagonals)

(198, 4,Chapter)

(5, 198,Exercises)

(199, 199,how many entries can you fit in a 2-gigabyte in-memory table? Will that be enough for the)

(200, 4,Chapter)

(5, 1,Adversarial Search)

(0, 200,Exercises)

(201, 5,have the same knowledge of the current state)

(6, 201,relation and enumerating the members of the relation For example, if X1 and X2 both have)

(202, 202,Defining Constraint Satisfaction Problems)

(203, 203,about this one? With CSPs, once we find out that a partial assignment is not a solution, we can)

(204, 5,Chapter)

(6, 204,Defining Constraint Satisfaction Problems)

(205, 205,CONSTRAINTS)

(206, 5,Chapter)

(6, 206,Defining Constraint Satisfaction Problems)

(207, 207,Linear programming problems do this kind of optimization)

(208, 5,Chapter)

(6, 208,Constraint Propagation: Inference in CSPs)

(209, 209,be inserted in the queue only d times because Xi has at most d values to delete Checking)

(210, 5,Chapter)

(6, 210,Constraint Propagation: Inference in CSPs)

(211, 211,than applying arc consistency to an equivalent set of binary constraints There are more)

(212, 5,Chapter)

(6, 212,Constraint Propagation: Inference in CSPs)

(9, 0,I)

(9, 213,Of course, Sudoku would soon lose its appeal if every puzzle could be solved by a)

(214, 5,Chapter)

(6, 214,Backtracking Search for CSPs)

(215, 215,what order should its values be tried (ORDER-DOMAIN-VALUES)?)

(216, 5,Chapter)

(6, 216,Backtracking Search for CSPs)

(217, 217,seems that that assignment constrains its neighbors, NT and SA, so we should handle those)

(218, 5,Chapter)

(6, 218,Backtracking Search for CSPs)

(219, 219,SA fails, and its conflict set is (say) {WA, NT, Q} We backjump to Q, and Q absorbs)

(220, 5,Chapter)

(6, 220,Local Search for CSPs)

(221, 1,constraints violated by a particular value, given the rest of the current assignment)

(0, 221,observations from three weeks (!) to around 10 minutes)

(222, 5,Chapter)

(6, 222,The Structure of Problems)

(223, 223,Fortunately, we can do this (in the graph, not the continent) by fixing a value for SA and)

(224, 5,Chapter)

(6, 224,The Structure of Problems)

(225, 225,TREE WIDTH)

(226, 5,Chapter)

(6, 226,Summary)

(227, 227,computer science One of the most influential early examples was the SKETCHPAD sys-)

(228, 5,Chapter)

(6, 228,Bibliographical and Historical Notes)

(229, 229,the tree decomposition approach sketched in Section  Drawing on this work and on results)

(230, 5,Chapter)

(6, 230,Exercises)

(231, 231,the harder problem of solving them)

(232, 5,Chapter)

(6, 232,Exercises)

(233, 6,of such algorithms make the cycle cutset method practical?)

(7, 233,agent without going into any technical detail Then we explain the general principles of logic)

(234, 234,Knowledge-Based Agents)

(235, 235,agent is not an arbitrary program for calculating actions It is amenable to a description at)

(236, 6,Chapter)

(7, 236,The Wumpus World)

(237, 237,pen to be immutable—in which case, the transition model for the environment is completely)

(238, 6,Chapter)

(7, 0,PIT)

(4, 238,The Wumpus World)

(239, 239,relies on the lack of a percept to make one crucial step)

(240, 6,Chapter)

(7, 240,Logic)

(2, 0,PIT)

(2, 0,PIT)

(2, 0,PIT)

(2, 0,PIT)

(2, 0,PIT)

(2, 0,PIT)

(2, 0,PIT)

(2, 0,e)

(2, 0,PIT)

(2, 0,PIT)

(2, 0,PIT)

(2, 0,PIT)

(2, 0,PIT)

(2, 0,PIT)

(2, 0,PIT)

(2, 241,have ’orrible ’airy wumpuses in them)

(242, 6,Chapter)

(7, 242,Propositional Logic: A Very Simple Logic)

(243, 243,erything takes place, of course, in the wumpus world)

(244, 6,Chapter)

(7, 244,Propositional Logic: A Very Simple Logic)

(245, 245,can be computed with respect to any model m by a simple recursive evaluation For example,)

(246, 6,Chapter)

(7, 246,Propositional Logic: A Very Simple Logic)

(247, 247,complexity that is exponential in the size of the input)

(248, 6,Chapter)

(7, 248,Propositional Theorem Proving)

(249, 249,true in every model—which is essentially what the inference algorithm in Figure  does—)

(250, 6,Chapter)

(7, 250,Propositional Theorem Proving)

(251, 251,soning: changing one’s mind They are discussed in Section)

(252, 6,Chapter)

(7, 252,Propositional Theorem Proving)

(253, 253,clauses makes the resolution rule much cleaner, at the cost of introducing additional notation)

(254, 6,Chapter)

(7, 254,Propositional Theorem Proving)

(255, 255,This theorem is proved by demonstrating its contrapositive: if the closure RC(S) does not)

(256, 6,Chapter)

(7, 256,Propositional Theorem Proving)

(257, 257,proceeding The reader is encouraged to work through the example in detail)

(258, 6,Chapter)

(7, 258,Effective Propositional Model Checking)

(259, 259,logic This section can be skimmed on a first reading of the chapter)

(260, 6,Chapter)

(7, 260,Effective Propositional Model Checking)

(261, 261,appears most frequently over all remaining clauses)

(262, 6,Chapter)

(7, 262,Effective Propositional Model Checking)

(263, 263,take just two random guesses to find a model This is an easy satisfiability problem, as are)

(264, 6,Chapter)

(7, -1,The conjecture remains unproven)

(8, -1,Clause/symbol ratio m/n)

(8, 264,Agents Based on Propositional Logic)

(265, 265,if the time step (as supplied to MAKE-PERCEPT-SENTENCE in Figure ) is 4, then we add)

(266, 6,Chapter)

(7, 266,Agents Based on Propositional Logic)

(267, 267,Exercise  asks you to write out axioms for the remaining wumpus world fluents)

(268, 6,Chapter)

(7, 268,Agents Based on Propositional Logic)

(269, 269,increasingly expensive as the history gets longer)

(270, 6,Chapter)

(7, 270,Agents Based on Propositional Logic)

(271, 271,at each time up to some maximum time t;)

(272, 6,Chapter)

(7, 272,Agents Based on Propositional Logic)

(273, 273,the successor-state axioms)

(274, 6,Chapter)

(7, 274,Bibliographical and Historical Notes)

(275, 275,connectives The use of truth tables for defining connectives is due to Philo of Megara The)

(276, 6,Chapter)

(7, 276,Bibliographical and Historical Notes)

(277, 277,that all NP-hard problems have a phase transition Crawford and Auton (1993) located the)

(278, 6,Chapter)

(7, 278,Exercises)

(279, 279,contain a wumpus Following the example of Figure , construct the set of possible worlds)

(280, 6,Chapter)

(7, 280,Exercises)

(281, 281,This exercise looks into the relationship between clauses and implication sentences)

(282, 6,Chapter)

(7, 282,Exercises)

(283, 283,Give a trace of the execution of DPLL on the conjunction of these clauses)

(284, 6,Chapter)

(7, 7,try it out in the wumpus world Does it improve the performance of the agent?)

(8, 284,were developed precisely to provide a more general, domain-independent way to store and)

(286, 7,Chapter)

(8, 286,Representation Revisited)

(287, 287,reported 41mph for the same cars in the same movie)

(288, 7,Chapter)

(8, 288,Representation Revisited)

(289, 289,a large city” might be true in our world only to degree  in fuzzy logic)

(290, 7,Chapter)

(8, 290,Syntax and Semantics of First-Order Logic)

(291, 291,A model containing five objects, two binary relations, three unary relations)

(292, 7,Chapter)

(8, 292,Syntax and Semantics of First-Order Logic)

(293, 293,shown by a gray arrow Within each model, the related objects are connected by arrows)

(294, 7,Chapter)

(8, 294,Syntax and Semantics of First-Order Logic)

(295, 295,INTERPRETATION)

(296, 7,Chapter)

(8, 296,Syntax and Semantics of First-Order Logic)

(297, 297,The same meaning can be expressed using equality statements)

(298, 7,Chapter)

(8, 298,Syntax and Semantics of First-Order Logic)

(299, 299,DOMAIN CLOSURE)

(300, 7,Chapter)

(8, 300,Using First-Order Logic)

(301, 301,according to nature’s design))

(302, 7,Chapter)

(8, 302,Using First-Order Logic)

(303, 303,than of first-order logic The importance of this distinction is explained in Chapter 9)

(304, 7,Chapter)

(8, 304,Using First-Order Logic)

(305, 305,For example, we have)

(306, 7,Chapter)

(8, 306,Knowledge Engineering in First-Order Logic)

(307, 307,might or might not need to take into account stray capacitances and skin effects)

(308, 7,Chapter)

(8, 308,Knowledge Engineering in First-Order Logic)

(309, 0,gates Signals flow along wires to the input terminals of gates, and each gate produces a)

(2, 309,AND gates, and one OR gate)

(310, 7,Chapter)

(8, 310,Knowledge Engineering in First-Order Logic)

(311, 311,Gate(O1) ∧Type(O1) = OR)

(312, 7,Chapter)

(8, 312,Summary)

(313, 313,identical to Frege’s Oddly enough, Peano’s axioms were due in large measure to Grassmann)

(314, 7,Chapter)

(8, 314,Exercises)

(315, 315,domains are allowed Give at least two examples of sentences that are valid according to the)

(316, 7,Chapter)

(8, 316,Exercises)

(317, 317,why it is incomplete compared to Equation (), and supply the missing axiom)

(318, 7,Chapter)

(8, 318,Exercises)

(319, 319,g There is an agent who sells policies only to people who are not insured)

(320, 7,Chapter)

(8, 320,Exercises)

(321, 8,l Joe owns a copy of every album on which all the songs are sung by Billie Holiday)

(9, 321,folkloric axiom stating that all greedy kings are evil:)

(322, 322,Propositional vs First-Order Inference)

(323, 323,interpretation maps a variable to an object in the domain)

(324, 8,Chapter)

(9, 324,Unification and Lifting)

(325, 325,MODUS PONENS)

(326, 8,Chapter)

(9, 326,Unification and Lifting)

(327, 327,knowledge base and FETCH(q) returns all unifiers such that the query q unifies with some)

(328, 8,Chapter)

(9, 328,Unification and Lifting)

(329, 329,been the subject of intensive study and technology development)

(330, 8,Chapter)

(9, 330,Forward Chaining)

(331, 331,symbols, the proof of completeness is fairly easy We begin by counting the number of)

(332, 8,Chapter)

(9, 332,Forward Chaining)

(333, 333,are owned by Nono)

(334, 8,Chapter)

(9, 334,Forward Chaining)

(335, 335,the set of rules in the knowledge base to construct a sort of dataflow network in which each)

(336, 8,Chapter)

(9, 336,Backward Chaining)

(337, 337,chaining is used in logic programming systems)

(338, 8,Chapter)

(9, 338,Backward Chaining)

(339, 339,are “proved” by executing code rather than doing further inference For example, the)

(340, 8,Chapter)

(9, 340,Backward Chaining)

(341, 341,bindings of the variables)

(342, 8,Chapter)

(9, 342,Backward Chaining)

(343, 343,closed world assumption says that the only sentences that are true are those that are entailed)

(344, 8,Chapter)

(9, 344,Resolution)

(345, 345,when written with a right-to-left implication symbol (Kowalski, 1979)) and is often much easier to read)

(346, 8,Chapter)

(9, 346,Resolution)

(347, 347,empty clause The algorithmic approach is identical to the propositional case, described in)

(348, 8,Chapter)

(9, 348,Resolution)

(349, 349,been standardized apart)

(350, 8,Chapter)

(9, 350,Resolution)

(351, 351,Now that we have established that there is always a resolution proof involving some)

(352, 8,Chapter)

(9, 352,Resolution)

(353, 353,equals for equals in any predicate or function So we need three basic axioms, and then one)

(354, 8,Chapter)

(9, 354,Resolution)

(355, 355,For example, one can use the negated query as the set of support, on the assumption that the)

(356, 8,Chapter)

(9, 356,Summary)

(357, 357,brand (1930)) that are used to move quantifiers to the front of formulas Skolem constants)

(358, 8,Chapter)

(9, 358,Bibliographical and Historical Notes)

(359, 359,VAMPIRE (Riazanov and Voronkov, 2002) The COQ system (Bertot et al, 2004) and the E)

(360, 8,Chapter)

(9, 360,Exercises)

(361, 361,inference engine will allow the conclusion ∃y P(q, q) to be inferred from the premise)

(362, 8,Chapter)

(9, 362,Exercises)

(363, 363,e Write a faster sorting algorithm, such as insertion sort or quicksort, in Prolog)

(364, 8,Chapter)

(9, 364,Exercises)

(365, 9,quences of a set of sentences Can any algorithm do this?)

(10, 365,for all four agent orientations, T time steps, and n2 current locations)

(366, 366,Definition of Classical Planning)

(367, 367,more restricted than PDDL: STRIPS preconditions and goals cannot contain negative literals)

(368, 9,Chapter)

(10, 368,Definition of Classical Planning)

(369, 369,Load(C2, P2, JFK ), Fly(P2, JFK , SFO), Unload(C2, P2, SFO)])

(370, 9,Chapter)

(10, 370,Definition of Classical Planning)

(371, 371,when y = Table, it has the precondition Clear(Table), but the table does not have to be clear)

(372, 9,Chapter)

(10, 372,Algorithms for Planning as State-Space Search)

(373, 373,inverse of the actions to search backward for the initial state)

(374, 9,Chapter)

(10, 374,Algorithms for Planning as State-Space Search)

(375, 375,initial state, so we are done)

(376, 9,Chapter)

(10, 376,Algorithms for Planning as State-Space Search)

(377, 377,in a goal or precondition with a new positive literal, P ′)

(378, 9,Chapter)

(10, 378,Planning Graphs)

(379, 379,ables As we mentioned on page 368, it is straightforward to propositionalize a set of ac-)

(380, 9,Chapter)

(10, 380,Planning Graphs)

(381, 381,which gi first appears in the planning graph constructed from initial state s We call this the)

(382, 9,Chapter)

(10, 382,Planning Graphs)

(383, 383,one has the effect At(Spare, Ground) and the other has its negation)

(384, 9,Chapter)

(10, 384,Planning Graphs)

(385, 385,failure because there is no possibility of a subsequent change that could add a solution)

(386, 9,Chapter)

(10, 386,Other Classical Planning Approaches)

(387, 2007,Winning Systems (approaches))

(2008, 2007,GAMER (model checking, bidirectional search))

(2008, 2005,LAMA (fast downward search with FF heuristic))

(2006, 2005,SATPLAN, MAXPLAN (Boolean satisfiability))

(2006, 2003,SGPLAN (forward search; partitions into independent subproblems))

(2004, 2003,SATPLAN (Boolean satisfiability))

(2004, 2001,FAST DIAGONALLY DOWNWARD (forward search with causal graph))

(2002, 2001,LPG (local search, planning graphs converted to CSPs))

(2002, 1999,TLPLAN (temporal action logic with control rules for forward search))

(2000, 1999,FF (forward search))

(2000, 1997,TALPLANNER (temporal action logic with control rules for forward search))

(1998, 387,the variable with a disjunction over constants For example, the goal of having block A)

(388, 9,Chapter)

(10, 388,Other Classical Planning Approaches)

(389, 389,Ai(x,   ) ̸= Aj(y,   ))

(390, 9,Chapter)

(10, 390,Other Classical Planning Approaches)

(391, 391,humans to understand what the planning algorithms are doing and verify that they are correct)

(392, 9,Chapter)

(10, 392,Summary)

(393, 393,gorithmic approach; what we call the “classical” language is close to what STRIPS used)

(394, 9,Chapter)

(10, 394,Bibliographical and Historical Notes)

(395, 395,for proving properties of binary decision diagrams, including the property of being a solution)

(396, 9,Chapter)

(10, 396,Exercises)

(397, 397,place to place, pushing movable objects (such as boxes), climbing onto and down from rigid)

(398, 9,Chapter)

(10, 398,Exercises)

(399, 399,as the action schema)

(400, 9,Chapter)

(10, 10,c Discuss how one might modify a satisfiability algorithm such as WALKSAT so that it)

(11, 400,ral information is added to the plan to ensure that it meets resource and deadline constraints)

(402, 10,Chapter)

(11, 402,Time, Schedules, and Resources)

(403, 403,LS(A) = minB ≻A LS(B) −Duration(A))

(404, 10,Chapter)

(11, 29,AddEngine1)

(30, 29,AddWheels1)

(10, 9,Finish)

(10, 14,Inspect2)

(15, 59,AddWheels2)

(60, -1,AddEngine2)

(90, 404,Time, Schedules, and Resources)

(405, -1,AddEngine2)

(120, 405,total completion time of a plan This is currently an active area of research)

(406, 10,Chapter)

(11, 406,Hierarchical Planning)

(407, 407,the basic concepts of hierarchical planning)

(408, 10,Chapter)

(11, 408,Hierarchical Planning)

(409, 409,are constructed, eliminating detail that is specific to the problem instance (eg, the name of)

(410, 10,Chapter)

(11, 410,Hierarchical Planning)

(411, 411,represented As with the classical action schemas of Chapter 10, we represent the changes)

(412, 10,Chapter)

(11, 412,Hierarchical Planning)

(413, ,goal state, and the left-hand circled state is the intermediate goal obtained by regressing the)

